\documentclass[letterpaper,10pt]{amsart}
\usepackage[utf8]{inputenc}
\usepackage[left=1in,top=1in,right=1in,bottom=1in]{geometry}

%opening
\title{Rainfall--Runoff Game}
\author{ESCI 4701/8701: Geomorphology; Updated 2019, 2023}

\begin{document}

%\begin{flushright}
%Name: $\rule{6cm}{0.15mm}$
%\end{flushright}
%\vspace{1cm}

{\let\newpage\relax\maketitle}

\textit{The rainfalll--runoff game is a way to give you intuition into how rainfall and runoff move across the landscape and produce river discharge.}

\section{Backstory}

A rainfall event just covered the landscape in a certain depth of water. You are a hydrologist who is working to simulate the resultant river discharge based on a set of simple rules and a gameboard that represents your watershed.

Play this game \textbf{in pairs}. You will play multiple times, taking turns with who is setting up the board and who is helping to count and follow the rules.

\section{Materials}

\begin{enumerate}
 \item Pen or pencil
 \item (optional) Markers of different colors to create your game board
 \item One hex-grid paper for each game that you will play
 \item Scratch paper or a note-taking device to record game configuration as well as discharge through time at the river outlet
\end{enumerate}

\section{Setting up the board}

\begin{enumerate}
 \item Write both team members' names on the board.
 \item Draw the outline of a drainage basin on the hex paper, making sure to draw around the edges of full hexes. Standard example drainage basins have a tear-drop shape, but real drainage basins come in many shapes.
 \item Draw one or a few lines to represent rivers.
 \begin{itemize}
 	\item These lines must go from one hex center to another
 	\item They must form a convergent network
 	\item Consider the minimum drainage area at which a river forms; ensure that this is approximately consistent across your landscape.
 	\item If you need inspiration, study real river networks, perhaps in Google Maps topographic view or on other maps that you find online.
 \end{itemize}
% \item Place a number of game counters in each cell proportional to how heavy the rain was.
% \begin{itemize}
% 	\item Place them red-side up for surface water.
% 	\item You may choose for rainfall to cover your entire watershed or just a portion of it, based on what scenario you are simulating. Typically, we conver the whole watershed.
% \end{itemize} 
\end{enumerate}

\section{Gameplay}

\subsection{Setup}

For your first play-through, we will simulate a ``unit hydrograph'': A rainstorm of a single ``unit'' size that covers the entire watershed uniformally. Later games may involve different initial conditions.

\begin{enumerate}
 \item Begin by placing one counter \textbf{red}-side up in each cell. The red side indicates that this is surface water -- the rainfall that has just fallen.
 \item Decide on the subsurface storage capacity of the cells. The default choice is 1. Write this on your map.
 \item Decide on the infiltration rate. The default choice is 1. Write this on your map.
\end{enumerate}

\newpage

\subsection{In each round}
\begin{enumerate}
 \item \textbf{Streamflow.} Move all counters on river cells 5 cells downstream. If these counters exit the map, save them in a ``discharge'' pile.
 \item \textbf{Infiltration.} In each cell that contains surface water (red) is not already at its subsurface-water storage capacity, flip a number of surface-water pieces equal to the \textbf{infiltration rate} so their \textbf{yellow} sides are showing. This indicates water that this water has infiltrated into the subsurface. If infiltration would exceed the storage capacity of the cell, only infiltrate enough water (flip enough game pieces) as the cell can hold.
 \item \textbf{Exfiltration.} If a cell contains more water than it should be able to hold, flip all of these game pieces in excess of the storage capacity so their red (surface-water) sides are showing. This excess water exceeds the subsurface storage capacity and exfiltrates.
 \item \textbf{Overland flow.} Move the surface water counters \textit{two cells} towards the nearest river. Do this in order from most downstream to most upstream; this is in order to make sure that you do not accidentally move a counter twice. If a counter reaches a river on the first move, move it two additional spaces downstream, for a total of three cells. Save any of these counters that exit the map in the ``discharge'' pile.
 \item \textbf{Subsurface flow.} Again, starting from the most downstream counters and moving to the most upstream ones, move all of the counters representing shallow subsurface flow towards the nearest river. These counters move \textit{one cell} only.
 \item \textbf{Discharge.} Count how many counters exited the map during this round and record this number alongside the game-round number. Then, place these game pieces in a discard pile. These data eventually will form your hydrograph.
\end{enumerate}

Repeat this until all of the counters have exited your map.

After your first game, generate an experiment to test some aspect of the hydrological system and how it might affect river discharge. Options include:
\begin{itemize}
	\small
	\item Changing the number of channels (i.e., the \textit{drainage density}).
	\item Changing the shape of the drainage basin.
	\item Changing the subsurface storage capacity, for example, to two counters of water -- this represents soils that can hold more water.
	\item Changing the infiltration rate -- the number of game pieces (water) that can infiltrate during each round.
	\item Adding more rainfall -- two or three counters per cell, for example, could indicate a larger initial storm.
	\item Adding additional water during several rounds, for example, to indicate a longer duration storm. Will you eventually reach a steady state (water out = water in)?
	\item Adding rainfall to only part of the map (e.g., for an isolated storm).
\end{itemize}
Each group will need to play at least three games.
%You will need to run at least two games of your own to answer the questions below. I encourage you to work in teams and strategize with partners to analyze each other's hydrographs as well. This will give you the ability to test more of these bullet-point items and answer Question 2.

\section{Assignment (35 points)}

\textit{You may share your materials for submission, but each student will need to submit their own work and write-up. Include the name of your partner with your write up.}

\begin{enumerate}
 \item \textbf{(15 points)} Plot three or more hydrographs with discharge (counters) vs. time (game rounds). Beneath each hydrograph, include a photo of the board layout and information on the options chosen. \textit{You will also be graded on the realism of your watersheds; you may be creative, but they must be plausible.}
 \item \textbf{(10 points)} Describe which experiment you ran and what you found. Discuss what you learned in a hydrological context. A detailed interpretation that implies knowledge of watershed processes is expected.
 \item \textbf{(5 points)} Tile drains to rivers in the Midwest now make shallow subsurface water flow move much more quickly. Explain how this should shift and/or otherwise alter the flood peak, and if you would like, play the game again (and record the results) by allowing all subsurface flow to follow surface-water flow rules. This latter step is not required, but will ensure that you have a reasonable answer to this question (and should be quick, as it simplifies gameplay).
 \item \textbf{(5 points)} Based on our discussions of threshold shear stresses for sediment transport and erosion, which kinds of floods do you think might be most important for landscape evolution?
\end{enumerate}

\end{document}
